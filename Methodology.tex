%%%% Time-stamp: <2012-08-20 17:41:39 vk>

%% example text content
%% scrartcl and scrreprt starts with section, subsection, subsubsection, ...
%% scrbook starts with part (optional), chapter, section, ...
%%\chapter{Example Chapter}

\chapter{Methodology}
Our approach to the Visual Localization problem is outlined broadly by these steps:
\begin{itemize}
	\item Convert pointcloud of the urban scene into a representation with building contours as seen from the road. This is the \emph{query} map tile.\\
	\item Upon a dataset of overlapping map tiles of an area, train a Pose Regression Network (MapNet) to predict a 2d translation from a reference map tile. \\
	\item Infer the location of the \emph{query} map tile directly upon the map raster on which MapNet was trained. \\  
\end{itemize}

\myfig{thesis/our_pipeline}%% filename
{scale=0.5}%% width/height
{Our Pipeline}%% caption
{Our Pipeline}%% optional (short) caption for list of figures
{tb4.1} % From OpenCV Tutorials.

The approach can be characterized as a hybrid Pose Regression, wherein the query image is a map-tile like representation.


\section{MapNet}

%% vim:foldmethod=expr
%% vim:fde=getline(v\:lnum)=~'^%%%%\ .\\+'?'>1'\:'='
%%% Local Variables: 
%%% mode: latex
%%% mode: auto-fill
%%% mode: flyspell
%%% eval: (ispell-change-dictionary "en_US")
%%% TeX-master: "main"
%%% End: 
