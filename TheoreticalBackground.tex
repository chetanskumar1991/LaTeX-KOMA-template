%%%% Time-stamp: <2012-08-20 17:41:39 vk>

%% example text content
%% scrartcl and scrreprt starts with section, subsection, subsubsection, ...
%% scrbook starts with part (optional), chapter, section, ...
%%\chapter{Example Chapter}

\chapter{Theoretical Background}
%\graphicspath{ {./figures/thesis/} }

\section{The Basics of Structure from Motion}
In this section, we review the fundamental principles of the SfM pipeline. The SfM method is the classical solution to the visual localization problem, and the diagram below outlines the basic steps of the SfM pipeline.

 \myfig{thesis/simplified_sfm_pipeline}%% filename
       {scale=0.4}%% width/height
       {Simplified SfM Pipeline}%% caption
       {Basic SfM Pipeline}%% optional (short) caption for list of figures
       {tb2.1}

We shall describe the basic ideas of each stage of the pipeline, as relevant to the localization problem. 

\subsection{Camera Calibration}
Consider the case of projecting a point in 3D to a the image 2D plane, outlined in the figure below. The 2D point $\mathbf{x}$ can be described by the ray $\lambda[u\ v\ 1]^T$ in projective space.

The 3D point $\mathbf{x}$ can be projected onto the image plane by a projection matrix $\mathbf{P}$, thusly:

\[\mathbf{x} = \mathbf{PX} = \mathbf{K[R|t]X}\]

Note the decomposition of the projection matrix $\mathbf{P}$ as $\mathbf{K[R|t})$, wherein \textbf{K} is called the \emph{ intrinsic matrix}. $\mathbf{[R|t]}$ is a concatenation of the rotation and translation of the camera w.r.t some world coordinate system, called the \emph{extrinsic matrix}.

\myfig{thesis/pinhole_camera}%% filename
{scale=0.6}%% width/height
{Pinhole Camera}%% caption
{Pinhole Camera}%% optional (short) caption for list of figures
{tb2.2}

We model \textbf{K} as follows:

\[\mathbf{K} = \begin{bmatrix}
f & s & c_x\\
0 & af & c_y\\
0 & 0 & 1
\end{bmatrix}\]

$f$ is the focal length, and $c_x, c_y$ is the principal point. $s$ is the shear angle between the axes, and a is the aspect ratio. Multiplying a homogenous coordinate in 3-space will clearly lead to a shift by the principal point, and scaling by $f$ upon perspective division. 

The aim of Camera Calibration is to recover the intrinsics and/or extrinsics. If we denote the 3D points as $\mathbf{X}$, and the corresponding 2D points as $\mathbf{x}$, we can formulate the following set of equations:

\[\mathbf{x}=\begin{bmatrix}
u\\
v\\
1\end{bmatrix}, \mathbf{P} = \begin{bmatrix}
\mathbf{P_1^T}\\
\mathbf{P_2^T}\\
\mathbf{P_3^T}\end{bmatrix} \]

where $u, v$ are the coordinates of the 2D point, and $\mathbf{P_i^T}$ are transposes of the columns of $\mathbf{P}$. 

\[\mathbf{x} = \mathbf{PX}\]

\[\lambda\begin{bmatrix}
u\\
v\\
1\end{bmatrix} = \begin{bmatrix}
\mathbf{P_1^T}\\
\mathbf{P_2^T}\\
\mathbf{P_3^T}\end{bmatrix}\mathbf{X}\]

solve for $\lambda$, and we can write the above as:

\[\mathbf{P_3^TX}u = \mathbf{P_1^TX}\]
\[\mathbf{P_3^TX}v = \mathbf{P_2^TX}\]

Writing this system of equations as a postmultiplication of the $\mathbf{P_i^T}$'s gives us,

\[\mathbf{0} = \underbrace{\begin{bmatrix}
\mathbf{X^T} & 0 & -\mathbf{X^T}u\\
0 & \mathbf{X^T} & -\mathbf{X^T}v\\
\end{bmatrix}}_{\text{\textbf{A}}} \begin{bmatrix}
\mathbf{P_1}\\
\mathbf{P_2}\\
\mathbf{P_3}\end{bmatrix}\]

If we solve the above equation for more than 6 points, we get more than 12 constraints (each equation contributes 2 constraints). The decomposition to intrinsic and extrinsic components can be acquired by an RQ decomposition, as the rotation \textbf{R} is an orthonormal matrix.

Distortions due to actual lens shapes can be modeled as a simple radial distance-based displacement of the 2d points, as follows:

\[\bar{\mathbf{x}} = [u\ v]^T (1 + k_1r^2 + k_2r^2 + ...)\]

For accurate SfM, it is crucial to get a good estimation of the camera projection matrix. 

\section{Feature Extraction}











 



 





%% vim:foldmethod=expr
%% vim:fde=getline(v\:lnum)=~'^%%%%\ .\\+'?'>1'\:'='
%%% Local Variables: 
%%% mode: latex
%%% mode: auto-fill
%%% mode: flyspell
%%% eval: (ispell-change-dictionary "en_US")
%%% TeX-master: "main"
%%% End: 
