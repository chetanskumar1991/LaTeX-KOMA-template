%%%% Time-stamp: <2012-08-20 17:41:39 vk>

%% example text content
%% scrartcl and scrreprt starts with section, subsection, subsubsection, ...
%% scrbook starts with part (optional), chapter, section, ...
%%\chapter{Example Chapter}

\chapter{Theoretical Background}
%\graphicspath{ {./figures/thesis/} }

\section{The Basics of Structure from Motion}
In this section, we review the fundamental principles of the SfM pipeline. The SfM method is the classical solution to the visual localization problem, and the diagram below outlines the basic steps of the SfM pipeline.

 \myfig{thesis/simplified_sfm_pipeline}%% filename
       {scale=0.4}%% width/height
       {Simplified SfM Pipeline}%% caption
       {Basic SfM Pipeline}%% optional (short) caption for list of figures
       {tb2.1}

\subsection{Camera Calibration}


 





%% vim:foldmethod=expr
%% vim:fde=getline(v\:lnum)=~'^%%%%\ .\\+'?'>1'\:'='
%%% Local Variables: 
%%% mode: latex
%%% mode: auto-fill
%%% mode: flyspell
%%% eval: (ispell-change-dictionary "en_US")
%%% TeX-master: "main"
%%% End: 
